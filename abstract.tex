% Abstract

Plasma in the Earth's ionosphere is highly irregular on scales ranging between a few centimeters and hundreds of kilometers.  Small-scale irregularities or plasma waves can scatter radio waves resulting in a loss of signal for navigation and communication networks.  The polar region is particularly susceptible to strong disturbances due to its direct connection with the Sun's magnetic field and energetic particles.  In this thesis, factors that contribute to the production of decameter-scale plasma irregularities in the polar \(F\) region ionosphere are investigated.  Both global and local control of irregularity production are studied, i.e. we consider global solar control through solar illumination and solar wind as well as much more local control by plasma density gradients and convection electric field.

In the first experimental study, solar control of irregularity production is investigated using the Super Dual Auroral Radar Network (SuperDARN) radar at McMurdo, Antarctica.  The occurrence trends for irregularities are analyzed statistically and a model is developed that describes the location of radar echoes within the radar's field-of-view.  The trends are explained through variations in background plasma density with solar illumination affecting radar beam propagation.  However, it is found that the irregularity occurrence during the night is higher than expected from ray tracing simulations based on a standard ionospheric density model.  The high occurrence at night implies an additional source of plasma density and it is proposed that large-scale density enhancements called polar patches may be the source of this density.

The second study is concerned with modeling irregularity characterics near a large-scale density gradient reversal, such as those expected near polar patches, with a particular focus on the asymmetry of the irregularity growth rate across the gradient reversal.  Directional dependencies on the plasma density gradient, plasma drift, and wavevector are analyzed in the context of the recently developed general fluid theory of the gradient-drift instability.  In the ionospheric \(F\) region, the strongest asymmetry is found when an elongated structure is oriented along the radar's boresight and moving perpendicular to its direction of elongation.  These results have important implications for finding optimal configurations for oblique-scanning ionospheric radars such as SuperDARN to observe gradient reversals.

To test the predictions of the developed model and the general theory of the gradient-drift instability, an experimental investigation is presented focusing on decameter-scale irregularities near a polar patch and the previously uninvestigated directional dependence of irregularity characteristics.  Backscatter power and occurrence of irregularities are analyzed using measurements from the SuperDARN radar at Rankin Inlet, Canada, while background density gradients and convection electric fields are found from the north face of the Resolute Bay Incoherent Scatter Radar.  It is shown that irregularity occurrence tends to follow the expected trends better than irregularity power, suggesting that while the gradient-drift instability may be a dominant process in generating small-scale irregularities, other mechanisms such as a shear-driven instability or nonlinear process may exert greater control over their intensity.

It is concluded from this body of work that the production of small-scale plasma irregularities in the polar \(F\)-region ionosphere is controlled both by global factors such as solar illumination as well as local plasma density gradients and electric fields.  In general, linear gradient-drift instability theory describes small-scale irregularity production well, particularly for low-amplitude perturbations.  The production of irregularities is complex, and while ground-based radars are invaluable tools to study the ionosphere, care must be taken to interpret results correctly.
