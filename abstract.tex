% Abstract

Plasma irregularities in the ionosphere have been studied for decades using a variety of techniques.  A number of factors that contribute to the production and observation of plasma irregularities in the polar \(F\) region ionosphere are examined.  First, global factors related to both direct and indirect solar control are considered.  A band of ranges where echoes are statistically most likely to be observed is identified.  The location of this band varies diurnally and seasonally, but its motion relative to the radar can be explained by variations in the background plasma density effecting how far the beam must travel to be refracted to perpendicular to the magnetic field.  The high observed occurrence of nighttime backscatter implies an additional source of nighttime plasma density that is not accournted for by traditional models.  An example is presented that shows frequent polar patches may be the source of this density.  Backscatter occurrence peaks at the terminator.  Nighttime occurence is enhanced when a positive IMF By component is present, probably due to the convection pattern changing and creating a more favorable situation for gradient drift instability growth.

The gradient-drift instability is modeled within the field of view of a SuperDARN radar to identify when the growth rate is largest and examine asymmetry around a large-scale structure.  Dependencies on the relative directions of the gradient, plasma drift, and wavevector are all considered throughout the ionosphere.  In the \(F\) region, the strongest asymmetry is found when an elongated structure is oriented along the radar's boresight and moving perpendicular to its direction of elongation.  These results have implications for observations made with the SuperDARN network, but assume the directional dependencies predicted by linear GDI theory are applicable at decameter scales.

To test the accuracy of the predictions found from the model, irregularities surrounding a polar patch are analyzed in the context of linear GDI theory.  Backscatter power and occurrence from decameter scale irregularities are found using measurements from the SuperDARN radar at Rankin Inlet while background density gradients and electric fields are found from the north face of the Resolute Bay Incoherent Scatter Radar.  Particular emphasis is placed on directional dependence and wether small-scale irregularites show any signs of anisotropy.  While low powered echoes seem to follow the predictions of linear GDI theory, higher power echoes quickly exhibit much more complicated, nonlinear behavior.