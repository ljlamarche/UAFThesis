% Chapter 1: Introduction

\section{The Sun-Earth Environment}
The sun is both complex and highly dynamic.  The dynamics include a highly variable surface with strong magnetic fields energized plasma.  The surface can have sunspots, dark, relatively cool regions chracterized by intense magnetic fields, as well as coronal holes, which are low density areas that are a continuous source of plasma into the solar wind.  The intense magnetic fields associated with sunspots can break down releasing a large amount of energy and plasma, known as a solar flare.

The sun is always producing a continuous outflow of plasma known as the solar wind.  The solar wind typically has a density around 10 cm\(^{-3}\) at Earth and is traveling away from the sun at approximately 400 km/s.  However, there can be large and sudden outbursts of plasma from the sun such as solar flares or coronal mass ejections (CMEs) which can travel significanly faster.  The frequency of high energy events changes with the solar cycle, when the sun's activity changes regularly over a 22-year long cycle.  It takes 11 years for the sun to go from solar minimum, when solar activity is relatively low and there are few large events to solar maximum, when solar activity is generally high.

In addition to the outflow of plasma, the solar wind also consisted of the sun's magnetic field extended into the outer limits of the solar system, called the interplanetary magnetic field (IMF).  Because the sun has a short rotation period (about 11 days), the IMF gets twisted into a spiral, commonly called the Parker Spiral \textbf{CITE PARKER?}.  When the IMF reaches the Earth, the magnetic field is typically oriented at a \(45^\circ\) inclination relative to the ecliptic, but this is highly variable depending on solar wind conditions.

The IMF interacts with the Earth's own magnetic field, the magnetosphere.  The magnetosphere is approximately a dipole field, but dynamic pressure from the solar wind causes the sun facing side to be compressed and the rear side to be stretched into an extended tail.  In addition, magnetic reconnection can occur between the IMF and the magnetopause such that magnetic field lines that were closed in the magnetopause reconnect to magnetic field lines from the sun and become open to the solar wind.  This tends to happen in the polar regions and allows highly energized particles from the sun to flow down the magnetic field into the Earth's ionosphere.

\section{The Earth's Ionosphere}
\label{sec:ionosphere}
The Earth's ionosphere is a region of the upper atmosphere that ranges approximately between 50-1000 km in altitude where neutral gases have been excited so that they are partially ionoized, resulting in free electrons and ions.  Plasma is either partially or fully ionized gas.  When gases are heated, electrons are energized and removed from neutral particles, resulting in free ions and electrons.  Gases can either be partially ionized, in which only some of the molecules are ionized so charged particles are mixed with neutrals, or fully ionized such that all particles are ionized.

Two conflicting processes occure in the ionosphere to create a peak in electron density, referred to as the Chapman Layer \textbf{CTIE CHAPMAN?}.  Solar illumination and particle precipitation ionize neutrals, and the density of ions increase as altitude decreases because there are more neturals available to ionize.  However, below a certain point, the concentration of ions becomes large enough that recombination into neutral particles becomes a substantal factor, reducing the ion density.  The exact altitude where this density peak occurs is variable with time of day, location, season, and solar cycle.

\subsection{Ionosphere Regions}
\label{sec:ionosphere_regions}
The ionosphere has historically been divided into three distinct region.  The D region is located between 80 and 90 km in altitude and typically only exists in the daytime when the ionosphere is sunlit.  The E region typically exists between 90 and 120 km with a density peak between 105 and 110 km, depending on factors such as time of day and season.  Above 120 km is considered the F region, which has a density peak between 300 and 400 km.  The F region can actually have two peaks under some daytime conditions, notated as the F1 peak and the F2 peak.

Because the concentrations of ions and neutral particles changes with altitude, each of these regions have distinct plasma physical properties that make them unique and change how plasma structuring forms.  In the E region, electrons are magnetized so their motion is mostly due to gyration around the magnetic field.  However, the motion on ions is dominated by collisions between ions and neutrals, so ions in the E region are considered collisional.  In the F region where the neutral density is much lower, both ions and electrons are magnetized.  These differences vary the conductivity in both regions and can strongly impact how plasma waves develop, which will be discussed further in section \ref{sec:lit_experiment}.

\subsection{Plasma Structuring in the Polar Ionosphere}
\label{sec:polar_structure}
The polar ionosphere is a particularly interesting and dynamic region due to the presence of the magnetic poles and open field lines to the solar wind.  Plasma structuring can occur as variations in density, velocity, or temperature, but the main focus of this work will be density perturbations, often called irregularities.  Density irregularities in the polar cap can occur on scales ranging from thousands of kilometers to less than a centimeter.  Large-scale plasma irregularities are typically considered to be any density structuring on the scale of 1 -- 1000 km.  Some common examples are polar patches, polar holes, and sun-aligned arcs.  Polar patches are large density enhancements (usually at least twice the background plasma density) that travel across the polar cap with the background convection.  Polar holes are plasma density depletions that typically occur in the F region slightly poleward of the auroral oval.  Sun-aligned arcs are large density forms that stretch across the polar cap along the sun-earth line.  They are very narrow and often characterized by large and complex velocity shears on either side of the arc.

Intermediate-scale structuring typically ranges from 100 m -- 1 km.  This is also referred to as scintillation-causing structuring and is responsible for many of the negative space weather effects that are observed on Earth.  Radio signal scintillation through the ionosphere changes the phase and amplitude of the original signal, which can effect the ability that satelites have to communicate with ground recievers.  This can cause communication blackouts or introduce large errors into GPS calculations, effecting navigation.  Because communication and navigation technology is implemented in both every day civilian life as well as government and military intrests, the effects of this can be far reaching.

Small-scale structuring is generally considered to be all irregularities less than 100 m.  Although the effects of these scale sizes on navigation and communication systems are minimal, they can be the easiest to study because of large data sets collected from HF radars globally for several decades.  The irregularity sizes presented here are rough categories only and a high degree of coupling between different scales is generally considered reasonable.  For instance, large-scale structures such as polar patches may very well have internal intermediate- or small-scale structuring.  Some kind of turbulent cascade is often assumed to connect these different scales, but because various instruments or techniques usually only observe structuring on a particular scale, it is difficult to find direct evidence of this or how exactly it occurs.

\section{Observational Techniques and Models}
\section{SuperDARN}

%Figures
%- Map of radars (Northern and southern hemisphere)
%- Example convection map
%- Example ISR profile
%- Example MSISE profile?
%- Example of IGRF?
%- How does radar work diagram
%- Pulse sequence diagram

\section{CSR}
%-general introduction

Coherent Scatter Radars (CSR) have been used to study the Ionosphere for the last half century.  Radars are ground based instruments used to detect density structures in the ionosphere's plasma.  A basic radar consists of transmission and recieveing antenna.  The radar transmits a radio wave, which can reflect off a suititbly sized structure in the ionosphere.  Reflected waves can be detected by the recieving antennas.  The time difference between when the signal was transmitted and when it was detected  an then be used to determine how far away from the radar the target is.

The first instance of CSR being used to study plasma density structures in the ionosphere was the Scandanavian Twin Auroral Radar Experiment (STARE) in the 1970s and 1980s.  STARE consisted of two very-high frequency (VHF) radars with overlapping FoVs that were designed to measure FAIs in the E region of the ionosphere \citep{Greenwald1997}.  The advantage of two radars with overlaping FoVs was the ability to observe the same structures from two different orientations.  Because each radar can only measure the LoS velocity of a structure, two simultanious observations from different directions allows two different velocity vecotor compontents to be found, and hense the total velocity vector can be calculated.  This technique is still commonly used with CSR system.

There are three measurements typically made by CSR systems: backscatter power, line of sight (LoS) velocity, and spectral width.  Backscatter power is the power of the retuned signal that the radar's recievers detect.  Typically, a threshold is selected for the minimum power acceptable for a return to be considered an actual signal instead of background noise.  LoS velocity is the component of the total velocity that is measured along one beam of a radar.  Because radars measure velocity through the doppler shift of backscatter, only the component of the velocity vector that is along the radar beam can be measured.  Spectral width represents the width of the doppler power spectrum.
** EQUATIONS FOR LOS, WIDTH, ECT. **

%- Greenwald, 1985

CSR systems typically operate by transmitting pulses instead of continuously.  This allows the distance between the radar and the target to be found (based on the time difference between when the pulse is transmitted and recieved), but also limits the temporal resolution that the radar is capable of achieving.

%% INSERT RADAR PULSE DIAGRAM %%

The radar starts transmitting the pulse at \(t=-\Delta t\) and finishes at \(t=0\).  At \(t=?\), the pulse reaches the backscatter volume and begins reflecting off it.  At \(t=T-\Delta t\), the radar begins to received the the echoed signal, which finishs at \(t=T\).  The speed of the radar pulse in the Earth's atmosphere is approximated by the vacuum speed of light, \(c\).  The distance between the radar and the scattering volume is known as the range and is given in equation \ref{eqn:range}.
\begin{equation}
	\label{eqn:range}
	R = \frac{cT}{2}
\end{equation}
The size of the range gate is determined by the length of a radar pulse.  This is given in equation \ref{eqn:gatesize}.
\begin{equation}
	\label{eqn:gatesize}
	\Delta r = \frac{c\Delta t}{2}
\end{equation}
Longer pulses result in larger range gates, resulting in poorer spatial resolution.

When the radar transmits continuously, backscatter will be received, but because it is impossible to know when that signal was transmitted, the difference in time between transmission and receiving cannot be used to calculate how far away the backscatter volume is.  Conversly, if the radar only transmits a pulse of a certain length of a certain length and then "listens" for its return, the time between transmission and return is known and the distance the pulse traveled can be calculated and therefore the location of the backscattering volume.  However, another pulse cannot be transmitted until the first pulse returns, limiting the temporal resolution that can be achieved.  There are methods that can be used to improve this by allowing more than one pulse at a time to be sent and received, but the pulse method does inherently require a limited temporal resolution.  Increasing the length of a pulse can improve the temporal resolution, but decreases the spatial resolution.

More importantly than improving temporal resolution of data availabile, a short time between pulse returns is nessisary to allow dopler velocities to be calculated, particularly at long ranges.  The maximum dopler shifted frequency that can be observed by a radar is the Nyquist frequency, which is equivilent to half the sampling frquency.  **Explain where the Nyquist frequency comes from?**  Radars operating in single-pulse mode have a very limited sampiling frequency, as described above, particularly at long ranges where the pulse must travel a long distance.  This severely limits the dopler shift that can be measured, which places an upper limit on the LoS dopler velocity that can be calculated with this technique.  Using the frequency range of HF radars and the maximum range in the F-region ionosphere where backscatter would ideally be measured, single pulse mode results in a very low maximum doppler velocity, much lower than typical ionospheric flows.  **Example: Maximum doppler velocity that can be measured using single pulse at 2000 km range.**

%BENIFITS OF MULTIPULSE: Ponomarenko and Waters, 2006
Instead of waiting for each individual signal to be recieved before transmitting the next, HF radars typically employ a multipulse mode. In a multipulse mode, the radar transmits a series of pules with different timing between them.  
%AUTO CORRELATION FUNCTION

%- How to measure power, spectral width, LoS velocity
%	- What are these things?
%- Hanuise, 1993b
%- Baker, 1995

\subsection{SuperDARN}
- Overview
	- geography
	- number of radars
	- mid, auroral, high laditudes
	- history
		- STARE (Greenwald, 1978) - first use of CSR to study ionosphere, VHF radar
		- Goose Bay, built in 1983  (Greenwald, 1985) - first HF radar
	- measures power, spectral width, LoS velocity
	- decameter scale irregularities
	- 2 or more radars covering the same area can find the total velocity comonent
	- Network used to create convection maps
- Operational Specifications
	- Hardware - antennas?
	- Frequency
	- Modes
	- 8 pulse ACF (post 2011); 7 pulse before
		- 7 pulse - Greenwald 1983,1985; Farley 1972
		- 8 pulse - 2003 SDarn?
- How are convection maps produced?
	- Ruohoniemi et al. 1989
	- merge LoS measurments form 2+ radars with overlapping FoV (old method)
	- current method involves finding the "best fit" 2D ionospheric electrostatic potential
		- algorithm for this - Ruohoniemi and Baker, 1998
		- Legendre functions
		- Only use F-region measurements
		- Velocity data is supplimented with a statistical model when data point coverage from SuperDARN alone is insufficient

\section{Incoherent Scatter Radar}
Similar to CSRs, incoherent scatter radars (ISRs) are ground based radars that are used to probe the ionosphere.  However, they use a fundamentally different technique to observe structuring, which allows them to measure different things than CSRs.  Where as CSRs receive backscatter from large, "coherent" density structures in the ionosphere, ISRs operate by radar beams scattering due to the random thermal motion of electrons in the ionosphere.  The radar receivers then detect a spectra of frequencies with different power.  Analysis of this spectra will reveal the plasma frequency, and hence, the electron density, electron and ion temperature, and LoS velocity.
**CITE Gordon, 1958**

\section{Models}
Models can be useful tools for establishing the context of experimental measurments, especially in a global sense.  In particular, it is impractical to achieve full global coverage of all instruments, so models can be a useful way to fill in the gaps.  There are two main categories of models.  Theoretical models are based on first principles and global pictures are often found by running large computer simulations based on these first principles.  They can help give insight as to what causes specific phenomena because the process can be followed beginning to end.  Empirical models are based purely on measured data.  These can potentially be more accurate because they should replicate real measurments, but they also do not give any insight into how the system got to its ending state.  In addition, empirial models are often heavily averaged.
-empirical vs theory based models

\subsection{IRI}
The international reference ionosphere (IRI) is an empirical model of the Earth's ionosphere.  It was originally created in 1969 as a joint effort between the Comittee on Space Research (COSPAR) and the International Union of Radio Science (URSI) and has been periodically updated since then \citep{Rawer1978}.  The current version is IRI-2012 \citep{Bilitza2011}, which is used throughout the work done here.  It can be accessed either as an online module or by downloading and running the FORTRAN source code, both of which are available from the International Reference Ionosphere page on the NASA website (http://iri.gsfc.nasa.gov/).

The IRI model can produce outputs of electron density, temperatures of electrons, ions, and neutrals, ion composition (O+, H+, He+, O2+, NO+, N+), and total electron content (TEC).  The model requires the input of a location (latitude, longitude, and altitude) and time (year, date, and time).  Internally, the model has the additional capability of producing profiles in any of the input parameters.  This is particularly significant for height profiles, and allows the peak heights and densities of the E and F region to be calculated.

%- Empirical model of the global ionosphere
%- Parameters that can be calculated
%- How is it available?
%	- Online
%	- FORTRAN source code
%- Biltza, 1990

\subsection{MSISE}
The MSISE (Mass Spectrometer-Incoherent Scatter, Extended) model is an empirical model of the Earth's upper atmosphere.  It outputs overal neutral temperature and densities of particular species in the atmosphere, particularly O, N2, O2, He, Ar, H, and N.  Similar to the IRI model, input parameters include a location (latitude, longitude, height) and time (year, date, time), with the opportunity to produce profiles in any of those parameters.  The model was originally developed in 1986 to describe the neutral temperatures and densities above 100 km \citep{Batten1987}.  Currently, the 1990 rendition of the model is used (MSISE-90), which extends the lower altitudinal extent of the model and improves the predictions at higher altitudes \cite{Hedin1991}.
%- Empirical model of neutral atmosphere
%- parameters that can be calculated

\subsection{IGRF}
The international geomagnetic reference field (IGRF) is a model of the earth's magnetic field.  The model consists of mathematical expressions and corresponing coefficients that are updated regularly that represent both the Earth's magnetic field and how it changes over time \citep{Thebault2015}.  Because the Earth's magnetic field is in fact dynamic (the location of the geomagnetic poles can change by \(\sim\)10 km per year), the model constently needs to be updated not only to improve it but to keep it accurate with the present magnetic field.  It is the internationally agreed upon standard for the magnetic field strength and direction.
%- Earth's magnetic field
