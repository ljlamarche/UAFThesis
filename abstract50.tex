%% 50 word abstract


Plasma in the Earth's ionosphere is highly irregular on scales ranging
between a few centimeters and hundreds of kilometers. Small-scale
irregularities or plasma waves in the Earth's ionosphere can scatter radio waves resulting in a loss of
signal for navigation and communication networks. The polar region is
particularly susceptible to strong disturbances due to its direct connection with
the Sun's magnetic field and energetic particles. 

In this work, factors that
contribute to the production of small-scale plasma irregularities in the polar
F-region ionosphere are investigated. 

We present results from two
experimental studies using the Super Dual Auroral Radar Network and the
Resolute Bay Incoherent Scatter Radar facilities and one modeling study
utilizing recently developed general theory of the gradient-drift instability
(GDI). It is demonstrated that small-scale plasma irregularities in the polar
F-region exhibit a previously unreported dependence on the propagation
direction in accordance with directionally-dependent GDI theory, although
large-amplitude perturbations quickly become controlled by more complicated
processes. The production of irregularities is complex, and while ground-based
radars are invaluable tools to study the ionosphere, care must be taken to
interpret results correctly.





Plasma irregularities in the Earth's polar ionosphere can scatter radio waves resulting in disruptions to navigation/communication systems. It is demonstrated that small-scale irregularities depend on the propagation direction in accordance with gradient-drift instability theory, although large-amplitude perturbations are controlled by more complicated processes.
